\subsection{Robot movement}

	The robot being a two-wheeled-robot, the motion can be described by a forward movement speed ($\dot{y}$) and turning rate ($\dot{\varphi}$), which depend on the wheel rotations ($\dot{\theta}_i$):
	\begin{gather}
	\begin{bmatrix}
		\dot{y}\\\dot{\varphi}
	\end{bmatrix}=\vc{Z}
	\begin{bmatrix}
		\dot{\theta}_1\\			
		\dot{\theta}_2
	\end{bmatrix},\\
	\vc{Z}=R
	\begin{bmatrix}
		\frac{1}{2} & \frac{1}{2}\\ \frac{1}{w} & -\frac{1}{w}
	\end{bmatrix},
	\end{gather}
	where $R$ the radius of the wheels and $w$ is the width of the robot.
	
	Since the system is non-holonomic, to reach a certain target position the robot is turned in the direction of the target ($\delta\varphi$), and moved the distance between the current location and the target ($\delta y$). Therefore the commanded wheel rotations are:
	\begin{gather}
		\begin{bmatrix}
			\theta_1\\			
			\theta_2
		\end{bmatrix}=\vc{Z}^{-1}
		\begin{bmatrix}
			\delta y\\\delta \varphi
		\end{bmatrix}
	\end{gather}
	
	Since the hardware contains a low level PID controller for the actuator positions, this was enough to move the robot to the desired position.
		